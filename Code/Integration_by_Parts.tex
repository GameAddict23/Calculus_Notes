\documentclass{article}

\usepackage[letterpaper, left=1in, right=1in, top=1in, bottom=1in]{geometry}
\usepackage{amsmath, physics, enumitem, microtype, hyperref, fancyhdr}

\pagestyle{fancy}

\title{Integration by Parts}
\author{Laith Toom}
\date{20/2/2023}

\begin{document}
\maketitle
\newpage

\section{What is Integration by Parts?}
Let us say that we had an expression
\[ f(x)g(x) \]
and that expression was differeniated:
\[ \derivative{}{x}f(x)g(x) = f'(x)g(x) + f(x)g'(x) \tag{1}\label{eq1} \]
in order to get back to $f(x)g(x)$, we would need to integrate (obviously):
\begin{align*}
    \int \derivative{}{x}\left[f(x)g(x)\right] \,dx &= \int \left[f'(x)g(x) + f(x)g'(x)\right] \,dx \\
    f(x)g(x) &= \int f'(x)g(x) \,dx + \int f(x)g'(x) \,dx 
\end{align*}
We can now solve for the integrals:
\begin{align*}
    \int f'(x)g(x) \,dx \\
    \int f(x)g'(x) \,dx
\end{align*}
but we will primarily look to solve for the latter integral $\int f(x)g'(x)$.
Solving for that integral, we get the formula:
\[ \boxed{ \int f(x)g'(x) \,dx = f(x)g(x) - \int f'(x)g(x) \,dx } \tag{2}\label{formula1} \]

\subsection{Example}
We have an integral:
\[ \int x\cos(x) \,dx \]
We will assume this integral is in the form:
\[ \int f(x)g'(x) \,dx \]
thus we can use the formula for integration by parts:
\[ \int f(x)g'(x) \,dx = f(x)g(x) - \int f'(x)g(x) \,dx \]
which when plugging in our functions becomes:
\[ \int x\cos(x) \,dx = x\int \cos(x) \,dx - \int \derivative{}{x}[x]\cos(x) \,dx \]
evaluting this expression, we get:
\begin{subequations}
    \begin{align}
        \int x\cos(x) \,dx &= x\int \cos(x) \,dx - \int \derivative{}{x}[x]\cos(x) \,dx \\      
                           &= x\sin(x) - \int (1)\cos(x) \,dx \\
                           &= x\sin(x) - \int \cos(x) \,dx \\
                           &= x\sin(x) - \sin(x) + C
    \end{align}
\end{subequations}
thus, we find that:
\[ \boxed{ \int x\cos(x) \,dx = x\sin(x) - \sin(x) + C } \]

\subsubsection{Remarks}
In this example:
\[ f(x)g(x) = x\sin(x) \]
thus the expression that was differeniated in equation \ref{eq1} was $x\sin(x)$.

\end{document}