\documentclass{article}

\usepackage{microtype}
\usepackage{amsmath}
\usepackage{enumitem}
\usepackage{etoolbox}
\usepackage{pgfplots, relsize}
\usetikzlibrary{calc}
\usepgfplotslibrary{fillbetween}
\pgfplotsset{compat=newest}

\usepackage{tikz}
\usepackage{tikz-3dplot}
\usepackage[letterpaper, bottom=100pt]{geometry}
\usepackage{physics}
\usepackage{MnSymbol}
\usepackage{float}

\newcommand{\diff}[1]{\frac{\mathrm{d}#1}{\mathrm{d}t}}
\newcommand{\tvert}{\biggr\rvert_{t_1}^{t_2}}
\newcommand{\derv}{\frac{\mathrm{d}}{\mathrm{d}t}}
\newcommand{\ddx}{\frac{\mathrm{d}}{\mathrm{d}x}}
\newcommand{\dx}{\mathrm{d}x}
\newcommand{\dof}[1]{\mathrm{d}#1}
\newcommand{\dydx}[1]{\frac{\mathrm{d}#1}{\dx}}

\patchcmd\subequations
 {\theparentequation\alph{equation}}
 {\subequationsformat}
 {}{}

\newcommand{\subequationsformat}{\theparentequation.\arabic{equation}}

\author{Laith}
\title{Exponential Change and Separable Differential Equations}
\date{1/30/2023}

\begin{document}
\maketitle

\section{Exponential Change}
Let $y$ be the size of a population at time $t$.
\[\mathrm{y}(t)\]
If rate of change of $y$ is proportional to its size $y$; then
\[\diff{y} = ky\,,\,k>0 \text{  and the population is increasing.}\]
\[
    \Rightarrow \,\, \frac{1}{y}\diff{y} = k \,\,
    \Rightarrow \,\, \frac{1}{y} \dof{y} \,\,
    \Rightarrow \,\, \int\frac{1}{y}\dof{y} = \int k\dof{t} \,\,
    \Rightarrow \,\, \ln\abs{y}=kt+C 
\]

\[
    \ln\abs{y}=kt+C\rightarrow e^{\ln y} = e^{kt+C}
    = e^c\cdot e^{kt}=Ce^{kt}
\]

\[\abs{y} = Ce^{kt} \,\, \Rightarrow \boxed{y=\pm Ce^{kt}}\]

\begin{center}
\begin{tikzpicture}
    \begin{axis} 
        [
            scale=0.5,
            domain=-3:3,
            xtick, ytick,
            xmax = 3,
            ymin=0,
            axis lines = center
        ]
        \addplot[dashed, domain=-3:0] {2^x+1}; 
        \addplot[-stealth, domain=0:3] {2^x+1}; 

        \node[scale=2] at (0, 2^0+1) {.};
        \node[xshift=-10, above] at (0, 2^0+1) {$y_0$};
    \end{axis}
\end{tikzpicture}
\end{center}
\[y(0) = y_0\]

\newpage
\section{Separable Differential Equations}
Let $\dydx{y} = \mathrm{f}(x, y)$ where $y=\mathrm{y}(t)$.

$\mathrm{f}(x, y)$ is separable if $\mathrm{f}(x, y)$ can be expressed as $\mathrm{g}(x)\cdot \mathrm{h}(y)$:
\begin{align}
    &\dydx{y} = \mathrm{f}(x, y)=\mathrm{g}(x)\cdot \mathrm{h}(y) &\text{Multiply both sides by $\dx$}\\
    &\dof{y} = \mathrm{f}(x, y)\dx=\mathrm{g}(x)\cdot \mathrm{h}(y)\cdot\dx &\\
    &\dof{y} = \mathrm{g}(x)\cdot \mathrm{h}(y)\dx &\\
    &\frac{1}{\mathrm{h}(y)}\dof{y} = \mathrm{g}(x)\dx &
\end{align}
which gives us:
\[\int \frac{1}{\mathrm{h}(y)}\dof{y} = \int \mathrm{g}(x)\dx\]

\end{document}