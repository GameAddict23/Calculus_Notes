\documentclass{article}

\usepackage{microtype}
\usepackage{amsmath}
\usepackage{enumitem}
\usepackage{etoolbox}
\usepackage{pgfplots, relsize}
\usetikzlibrary{calc}
\usepgfplotslibrary{fillbetween}
\pgfplotsset{compat=newest}

\usepackage{tikz}
\usepackage{tikz-3dplot}
\usepackage[letterpaper, bottom=100pt]{geometry}
\usepackage{physics}
\usepackage{MnSymbol}
\usepackage{float}

\newcommand{\diff}[1]{\frac{\mathrm{d}#1}{\mathrm{d}t}}
\newcommand{\tvert}{\biggr\rvert_{t_1}^{t_2}}
\newcommand{\derv}{\frac{\mathrm{d}}{\mathrm{d}t}}
\newcommand{\ddx}{\frac{\mathrm{d}}{\mathrm{d}x}}
\newcommand{\dx}{\mathrm{d}x}
\newcommand{\dof}[1]{\mathrm{d}#1}
\newcommand{\dydx}[1]{\frac{\mathrm{d}#1}{\dx}}
% \newcommand{\dy_dx}[1]{\frac{\mathrm{d}#1}{\dx}}

\patchcmd\subequations
 {\theparentequation\alph{equation}}
 {\subequationsformat}
 {}{}

\newcommand{\subequationsformat}{\theparentequation.\arabic{equation}}

\author{Laith}
\title{Exponential Change and Separable Differential Equations}
\date{1/30/2023}

\begin{document}
\maketitle

\section{Exponential Change}
Let $y$ be the size of a population at time $t$.
\[\mathrm{y}(t)\]
If rate of change of $y$ is proportional to its size $y$; then
\[\diff{y} = ky\,,\,k>0 \text{  and the population is increasing.}\]
\[
    \Rightarrow \,\, \frac{1}{y}\diff{y} = k \,\,
    \Rightarrow \,\, \frac{1}{y} \dof{y} \,\,
    \Rightarrow \,\, \int\frac{1}{y}\dof{y} = \int k\dof{t} \,\,
    \Rightarrow \,\, \ln\abs{y}=kt+C 
\]

\[
    \ln\abs{y}=kt+C\rightarrow e^{\ln y} = e^{kt+C}
    = e^c\cdot e^{kt}=Ce^{kt}
\]

\[\abs{y} = Ce^{kt} \,\, \Rightarrow \boxed{y=\pm Ce^{kt}}\]

\begin{center}
\begin{tikzpicture}
    \begin{axis} 
        [
            scale=0.5,
            domain=-3:3,
            xtick, ytick,
            xmax = 3,
            ymin=0,
            axis lines = center
        ]
        \addplot[dashed, domain=-3:0] {2^x+1}; 
        \addplot[-stealth, domain=0:3] {2^x+1}; 

        \node[scale=2] at (0, 2^0+1) {.};
        \node[xshift=-10, above] at (0, 2^0+1) {$y_0$};
    \end{axis}
\end{tikzpicture}
\end{center}
\[y(0) = y_0\]

\newpage
\section{Separable Differential Equations}
Let $\dydx{y} = \mathrm{f}(x, y)$ where $y=\mathrm{y}(t)$.

$\mathrm{f}(x, y)$ is separable if $\mathrm{f}(x, y)$ can be expressed as $\mathrm{g}(x)\cdot \mathrm{h}(y)$:
\begin{align}
    &\dydx{y} = \mathrm{f}(x, y)=\mathrm{g}(x)\cdot \mathrm{h}(y) &\text{Multiply both sides by $\dx$}\\
    &\dof{y} = \mathrm{f}(x, y)\dx=\mathrm{g}(x)\cdot \mathrm{h}(y)\cdot\dx &\\
    &\dof{y} = \mathrm{g}(x)\cdot \mathrm{h}(y)\dx &\\
    &\frac{1}{\mathrm{h}(y)}\dof{y} = \mathrm{g}(x)\dx &
\end{align}
which gives us:
\[\int \frac{1}{\mathrm{h}(y)}\dof{y} = \int \mathrm{g}(x)\dx\]

\subsection{Examples}
\paragraph{\#9)} \[2\sqrt{xy}\dydx{y}=1\,;\,x,\,y > 0\]
\begin{subequations}
% \begin{gather}
%     \dydx{y} = \frac{1}{2\sqrt{xy}} \\ 
% \end{gather}
\hrule
\paragraph{\#20)}
\[\dydx{y} = xy+3x-2y-6\]
\begin{gather}
    \dydx{y} = x(y+3)-2(y+3) \\ 
    \dydx{y} = (x-2)(y+3) \\ 
    \dydx{y}\cdot{\dx} = (x-2)(y+3)\cdot{\dx} \\ 
    dy = (x-2)(y+3)\cdot{\dx} \\ 
    \frac{1}{y+3}dy = (x-2)\cdot{\dx} \\ 
    \int \frac{1}{y+3}\,dy = \int (x-2)\,\dx \\ 
    \boxed{\ln{\abs{y+3}} = \frac{1}{2}(x-2)^2+C}
\end{gather}

\hrule
\newpage
\paragraph{\#16)}
\[\sec{x}\dydx{y}=e^{y+\sin{x}}\]
\begin{gather}
   \frac{1}{\cos{x}}\dydx{y}=e^{y+\sin{x}} \\ 
   \Rightarrow \dydx{y} = \cos{x}\cdot e^{y+\sin{x}} \\
   \dydx{y} = \cos{x}(e^y)(e^{\sin{x}}) \\
   \Rightarrow \frac{1}{e^y}\dydx{y} = \cos{x}(e^{\sin{x}}) \\
   \frac{1}{e^y}\dydx{y} \cdot \dx = \cos{x}(e^{\sin{x}})\dx \\
   \frac{1}{e^y}\,dy = \cos{x}(e^{\sin{x}})\dx \\
   \int \frac{1}{e^y}\,dy = \int \cos{x}(e^{\sin{x}})\,\dx \\
   \int e^{-y}\,dy = \int (e^{\sin{x}})\cos{x}\,\dx \\
   \text{$u = -y$ $du = -dy \rightarrow -du = dy$ $\biggr\rvert$ $u = \sin{x}$ $du = \cos{x}dx$} \\
   \int e^{u}\,du = \int e^{u}\,du \\
   -e^{u} = e^{u}+k \\
   -e^{-y} = e^{\sin{x}}+k \\
\end{gather}
\end{subequations}
\hrule

\paragraph{\#} Show that each function $y=f(x)$ is a solution of the indicated DE.

\[2y'+3y=e^{-x}\]
\begin{gather*}
        a)y=e^{-x} \\
        b)y=e^{-x}+e^{\frac{-3}{2}x} \\
\end{gather*}
\hrule
\newpage

\paragraph*{Example} In a certain region, the population, $P(t)$, in
thousand of people $t$ year after census begin is approximated using 
an exponential growth model. The initial census showed $P_0=90$ and 
the population 2 years later was 120.
\begin{enumerate}[label=\alph*)]
    \item Find a formula for $P(t)$.
\end{enumerate} 
\begin{subequations}
\begin{gather}
    P(t) = P_0e^{kt}=P_0{(e^k)}^t \\
    P(0)=90,000 \\
    \Rightarrow P(t)=90,000e^{kt} \\
    \text{Two years later $t=2$ $P(2)=120,000$} \\
    120,000=90,000e^{kt} \\
    12 = 9e^{k\cdot 2} \\
    \frac{12}{9} = {(e^{k})}^2 \\
    \sqrt{\frac{12}{9}} = (e^{k}) \\
    e^k = \frac{2}{\sqrt{3}} \\
    P(t) = 90,000(\frac{2}{\sqrt3})^t
\end{gather}

\begin{enumerate}[label=\alph*)]
    \item Find the population after 4 years $\Rightarrow P(4)=90,000(\frac{2}{\sqrt{3}})^4$
\end{enumerate}
\begin{gather}
    P(4) = 90,000{((\frac{2}{\sqrt{3}})^2)}^2 \\
    P(4) = 90,000(\frac{4}{3})^2 \\
    P(4) = 90,000(\frac{16}{9}) \\
    P(4) = 10,000(16) = 160,000 \\
\end{gather}

\begin{enumerate}[label=\alph*)]
    \item 
\end{enumerate}

\end{subequations}
\end{document}