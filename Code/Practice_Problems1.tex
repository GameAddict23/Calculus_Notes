\documentclass{article}

% __Preamble__
\usepackage{microtype, amsmath, physics, enumitem, etoolbox, mathtools, fancyhdr}
\usepackage[letterpaper, top=1in, bottom=1in, left=1in, right=1in]{geometry}

% __Changing page style__
\pagestyle{fancy}
\fancyhf{}
\fancyhead{}     
\fancyhead[R]{\leftmark}     
\fancyfoot[C]{\hrule\vspace{1em}\thepage}


% __Changing Subequation format to number.number
\patchcmd{\subequations}
{\theparentequation\alph{equation}}
{\theparentequation.\alph{equation}}{}{}

% __Allows for (1.1.1) counting__
\makeatletter
\newcounter{parentsubequation}% Counter for ``parent equation''.
\newenvironment{subsubequations}{
  \refstepcounter{equation}
  \protected@edef\theparentsubequation{\theequation}
  \setcounter{parentsubequation}{\arabic{equation}}
  \setcounter{equation}{0}
  \def\theequation{\theparentsubequation\arabic{equation}}
  \ignorespaces
}{
  \setcounter{equation}{\arabic{parentsubequation}}
  \ignorespacesafterend
}
\makeatother

% __Load after changing subequation format to avoid conflict__
\usepackage{hyperref}

% __Title page__
\title{Practice Problem Set 1}
\author{Laith}
\date{17/2/2023}

\begin{document}
\maketitle
\newpage

% ______Exponential Models______
\section{Exponential Models}
\begin{subequations}
\textbf{900 grams} of a radioactive material is brought into a lab.
The amount remaining after 4 days was 4/9 of what remained after 2 
days. Let $\mathrm{N}(t)$ be the number of grams of radioactive material.
\begin{enumerate}[label=(\alph*)]
    \item Find the number, in grams, of the remaining material $t$ years 
    after being brought into the lab.
    \item How many grams remain after 2 days?
    \item What is the half-life of the radioactive material?
    \item How many days will it takes for only 1 gram to remain? 
\end{enumerate}

\subsection{(a)}
\begin{subsubequations}
    The basic exponential model looks like this:
    \[ \mathrm{f}(t) = br^t \]
    where $b$ is the starting value, $r$ is the rate of change, and $t$ is 
    time.

    If $t_1 = 4 \:\mathrm{days}$ and $t_2 = 2 \:\mathrm{days}$, in which 
    the population at $t_1$ is $\frac{4}{9}$ of the population at $t_2$, we can 
    set up the following equation:
    \begin{align}
        &\mathrm{N}(t_1) = \frac{4}{9}\mathrm{N}(t_2) \\
        &\mathrm{N}(4) = \frac{4}{9}\mathrm{N}(2) \\
        &\mathrm{N}(4) = \frac{4}{9}\mathrm{N}(2) \\
        &br^4 = \frac{4}{9}br^2 \\
        &900r^4 = \frac{4}{9}900r^2 &\text{We know that $b=900$, however $b$ will cancel out.} \\
        \Rightarrow &r^4 = \frac{4}{9}r^2 &\text{We can take the square root of both sides:}\\
        &\sqrt{r^4} = \sqrt{\frac{4}{9}r^2} \Rightarrow r^2 = \frac{2}{3}r &\text{Divide by $r$.} \\
        \Rightarrow &\boxed{r = \frac{2}{3}}
    \end{align}
    Thus, our function $\mathrm{N}(t)$ is:
    \begin{align}
        &\mathrm{N}(t) = 900\left(\frac{2}{3}\right)^t
    \end{align}
    If we plug in $t=2$:
    \[ \mathrm{N}(2) = 900\left(\frac{2}{3}\right)^2 = 900\cdot \frac{4}{9} = 400 \]
    and if we plug in $t=4$:
    \[ \mathrm{N}(4) = 900\left(\frac{2}{3}\right)^4 = 900\cdot \frac{16}{81} = \frac{1600}{9} \]
    Using a calculator, dividing $\frac{1600}{9}$ by 400 should gives us $\frac{4}{9}$:
    \[ \frac{\frac{1600}{9}}{400} = \frac{1600}{9\cdot 400} = \frac{1600}{3600} = \frac{16}{36} = \frac{4}{9} \]

    Now that we our function, and since $t$ is in terms of days, we can multiply $t$ by 365
    (the average number of days per year) to get our answer for part (a):
    \[ \boxed{\mathrm{N}(t) = 900\left(\frac{2}{3}\right)^{365t}} \]
\end{subsubequations}

\subsection{(b)}
\begin{subsubequations}
    We can substitue 2 for $t$ to get our answer:
    \begin{align}
        &\mathrm{N}(2) = 900\left(\frac{2}{3}\right)^2 = 900 \cdot \frac{4}{9} = \boxed{400 \,\mathrm{g}}
    \end{align}
    Thus, \textbf{400 grams} of material remain after 2 days.
\end{subsubequations}

\subsection{(c)}
\begin{subsubequations}
    The half-life is basically how much time it takes for the population to reach half of 
    its original size, thus we are just solving for time $t$ when $\mathrm{N}(t)=\frac{1}{2}b$:
    \begin{align}
        &\mathrm{N}(t) = \frac{1}{2}900 = 450 \\
        &900\left(\frac{2}{3}\right)^t = 450 \\
        \Rightarrow &\left(\frac{2}{3}\right)^t = \frac{45}{90} = \frac{1}{2}
    \end{align}
    We can rewrite this using $\log$ of base $\frac{2}{3}$ since:
    \[ a^x = b \Rightarrow \log_a(b) = x \]
    \begin{align}
        &\log_{2/3}(\frac{1}{2}) = t \\
        &\log_{2/3}(1) - \log_{2/3}(2) = t \\
        &0 - \log_{2/3}(2) = t \\
        &-\log_{2/3}(2) = t \\
        \intertext{\centering Using a calculator, we find:}
        &\boxed{t \approx 1.71 \,\mathrm{days}}
    \end{align}
\end{subsubequations}

\subsection{(d)}
\begin{subsubequations}
    Now similar to part (c), we need to solve for time $t$, but this time:
    \[ \mathrm{N}(t) = 1 \]
    so to solve for $t$:
    \begin{align}
        &900\left(\frac{2}{3}\right)^t = 1 \\
        \Rightarrow &\left(\frac{2}{3}\right)^t = \frac{1}{900} \\
        &\log_{2/3}(1/900) = t \\
        &\log_{2/3}(1) - log_{2/3}(900) = t \\
        &-log_{2/3}(900) = t \\
        &\boxed{t \approx 16.78 \,\mathrm{days}} 
    \end{align}
\end{subsubequations}

\end{subequations}

\newpage
% ______Seperable Differential Equations________
\section{Separable Differential Equations}
\begin{subequations}
Find the explicit solution to these differential equations 
that satisfy the given initial condition:

\subsection*{(a)}
\begin{subsubequations}
    {\Large \[ (1+x^2)y' = \frac{1}{y^2};\quad y(0) = 2 \]}

    \begin{alignat}{2}
        (1+x^2)y' = \frac{1}{y^2}& &\implies\: &(1+x^2)\frac{dy}{dx} = \frac{1}{y^2} \\
        &&\implies\:  &(1+x^2)dy = \frac{1}{y^2}dx \nonumber \\
        &&\implies\:  &y^2(1+x^2)dy = dx \nonumber \\
        &&\implies\:  &y^2\,dy = \frac{1}{1+x^2}\,dx \\
        \intertext{Now to integrate both sides:}
        \int y^2\,dy = \int \frac{1}{1+x^2}\,dx& &\implies\: &\frac{y^3}{3}+C = \arctan(x)+C \\
        \intertext{We will assume that there is only one $C$:}
        &&\implies\: &\frac{y^3}{3} = \arctan(x)+C \nonumber \\
        &&\implies\: &y^3 = 3(\arctan(x)+C) \nonumber \\
        &&\implies\: &y = \sqrt[3]{3(\arctan(x)+C)} 
        \intertext{Now to solve for $C$ such that $y(0)=2$, in which $x=0$:}
        2 = \sqrt[3]{3(\arctan(0)+C)}& &\implies\: &2^3 = 3(\arctan(2)+C) \\
        &&\implies\: &8 = 3(\arctan(0)+C) \nonumber \\
        &&\implies\: &\frac{8}{3} = \arctan(0)+C \nonumber \\
        &&\implies\: &\frac{8}{3} = C
    \end{alignat}
    Thus, our equation is:
    \[ \boxed{y = \sqrt[3]{3\arctan(x)+\frac{8}{3}}} \]

\end{subsubequations}

\newpage
\subsection*{(b)}
\begin{subsubequations}
    {\Large\[x\,dy - (y + \sqrt{y})dx = 0\,;\quad y(1)=1\]}
    \begin{align}
        &x\,dy - (y + \sqrt{y})dx = 0 \\[0.5ex]
        \Rightarrow\: &x\,dy = (y + \sqrt{y})dx \\[0.5ex]
        \Rightarrow\: &x\frac{dy}{dx} = y + \sqrt{y} \\[0.5ex]
        \Rightarrow\: &\frac{dy}{dx} = \frac{y + \sqrt{y}}{x} \\[0.5ex]
        \Rightarrow\: &\frac{dy}{dx}\left(\frac{1}{y + \sqrt{y}}\right) = \frac{1}{x} \\[0.5ex]
        \Rightarrow\: &\frac{1}{y + \sqrt{y}}\,dy = \frac{1}{x}\,dx \\[0.5ex]
        \Rightarrow\: &\int \frac{1}{y + \sqrt{y}}\,dy = \int \frac{1}{x}\,dx \\[0.5ex]
        \Rightarrow\: &\int \frac{1}{\sqrt{y}(\sqrt{y} + 1)}\,dy = \int \frac{1}{x}\,dx \label{eq1} \\[0.5ex]
        \intertext{Integrating the left side:}
        &u = \sqrt{y}+1 \quad du = \frac{1}{2\sqrt{y}}dx \rightarrow dx = 2\sqrt{y}\,du \nonumber\\[0.5ex]
        \Rightarrow\: &\int \frac{1}{\sqrt{y}(u)}\,2\sqrt{y}\,du \nonumber\\[0.5ex]
        \Rightarrow\: &2\int \frac{1}{u}\,du \nonumber\\[0.5ex]
        \Rightarrow\: &2\left(\ln\abs{u}\right)+C \nonumber\\[0.5ex]
        \Rightarrow\: &2\left(\ln\abs{\sqrt{y}+1}\right)+C \nonumber\\[0.5ex]
        \intertext{Integrating right side:}
        \int &\frac{1}{x} = \ln\abs{x}+C \nonumber\\[0.5ex]
        \intertext{Returning to equation \ref{eq1}:}
        \Rightarrow\: &2\ln\abs{\sqrt{y}+1} + C = \ln\abs{x} + C \\[0.5ex]
        \intertext{We will get rid of the C on the left side since it is an 
        arbitrary constant, thus we will assume there is only one constant:}
        \Rightarrow\: &2\ln\abs{\sqrt{y}+1} = \ln\abs{x} + C \\[0.5ex]
        \intertext{Raise $e$ to the power of both sides in order to cancel out $\ln$:}
        \Rightarrow\: &e^{2\ln\abs{\sqrt{y}+1}} = e^{\ln\abs{x} + C} \Rightarrow e^{2\ln\abs{\sqrt{y}+1}} = e^{\ln\abs{x}}\cdot e^C \\[0.5ex]
        \Rightarrow\: &(\sqrt{y}+1)^2 = x\cdot e^C \nonumber \\[0.5ex]
        \Rightarrow\: &\sqrt{y}+1 = \sqrt{x\cdot e^C} \nonumber \\[0.5ex]
        \Rightarrow\: &\sqrt{y} = \sqrt{x\cdot e^C}-1 \nonumber \\[0.5ex]
        \Rightarrow\: &y = xe^C - 2\sqrt{xe^C} +1 \label{eq2}
        \intertext{Now to solve for $C$ such that the condition $y(1)=1$ is satisfied:}
        &1 = (1)\cdot e^C - 2\sqrt{(1)e^C} + 1 \\[0.5ex]
        \Rightarrow\: &1 = e^C - 2\sqrt{e^C} + 1 \nonumber \\[0.5ex]
        \Rightarrow\: &0 = e^C - 2\sqrt{e^C} \:\Rightarrow\: 2\sqrt{e^C} = e^C  \\[0.5ex]
        \Rightarrow\: &4e^C = {(e^C)}^2 \:\Rightarrow\: 4e^C = e^{2C}  \\[0.5ex]
        \Rightarrow\: &4 = e^C \:\Rightarrow\: \log_e(4) = C \\[0.5ex]
        \Rightarrow\: &\boxed{C = \ln(4)} 
        \intertext{Plugging this back into our original equation \ref{eq2}:}
        &y = xe^{\ln(4)} - 2\sqrt{xe^{\ln(4)}} + 1 \nonumber \\[0.5ex]
        &e^{\ln(a)}=a \nonumber \\[0.5ex]
        &y = 4x - 2\sqrt{4x} + 1 = 4x - 4\sqrt{x} + 1 \nonumber
        \intertext{Thus, our answer is:}
        &\boxed{y = 4x-4\sqrt{x}+1} 
    \end{align}
\end{subsubequations}

\subsection*{(c)}
\begin{subsubequations}
    {\Large \[\frac{dy}{dx} = \frac{y\ln(y)}{1+x^2}\,;\quad y(0)=e^2\]}

    \begin{alignat}{2}
        \frac{dy}{dx} = \frac{y\ln(y)}{1+x^2}& &\implies\: &\frac{1}{y\ln(y)}dy = \frac{1}{1+x^2}dx \\
        \intertext{Now we can integrate both sides:}
        \int \frac{1}{y\ln(y)}dy = \int \frac{1}{1+x^2}dx& \label{eq3}
        \intertext{Integrating left side:}
        u = \ln(y) \quad du = \frac{1}{y}dx \quad dx = y\,du \nonumber \\
        \int \frac{1}{y\ln(y)} dx& &\implies &\int \frac{1}{y\,u}y\,du \nonumber \\
        &&\implies &\int \frac{1}{u}du \nonumber \\
        &&\implies &\ln\abs{u} \implies \ln\abs{\ln(y)}
        \intertext{Integrating right side:}
        \int \frac{1}{1+x^2} = \arctan(x)+C
        \intertext{Returning to equation \ref{eq3}:}
        \ln\abs{\ln(y)} = \arctan(x)+C 
        \intertext{Raise $e$ to the power of both sides to cancel out $\ln$:}
        e^{\ln\abs{\ln(y)}}=e^{\arctan(x)+C}& &\implies &\ln(y) = e^{\arctan(x)+C} 
        \intertext{Repeat:}
        e^{\ln(y)}=e^{\left(e^{\arctan(x)+C}\right)}& &\implies &y = e^{\left(e^{\arctan(x)+C}\right)} \label{eq4}
        \intertext{Now to solve for $C$ such that $y(0)=e^2$:}
        e^2 = e^{\left(e^{\arctan(0)+C}\right)}& &\implies &e^2 = e^{\left(e^C\right)}
        \intertext{Since the bases are equal, we can assume that the exponents must also be equal, thus:}
        2 = e^C& &\implies &\ln(2) = C 
        \intertext{Thus, going back to equation \ref{eq4}, our equation is:}
        y = e^{\left(e^{\arctan(x)+\ln(2)}\right)}& &\implies &y = e^{\left(e^{\arctan(x)}e^{ln(2)}\right)} \nonumber \\
        &&\implies &\boxed{y = e^{\left(2e^{\arctan(x)}\right)}}
    \end{alignat}
\end{subsubequations}

\end{subequations}

\newpage
\section{Integration by Substituion}
\begin{subequations}
    
\subsection*{(a)}
\begin{subsubequations}
    {\Large \[ \int \frac{\left[\log_3(x)\right]^2}{x}\,dx \]}    

    \begin{alignat}{3}
        \int \frac{\left[\log_3(x)\right]^2}{x}\,dx& &\implies &\int \frac{1}{x}\left[\frac{\ln(x)}{\ln(3)}\right]^2\,dx \\[0.5ex]
        u = \ln(x) \quad du = \frac{1}{x}dx \quad dx = x\,du \\[0.5ex]
        &&\implies &\int \frac{1}{x}\left[\frac{u}{\ln(3)}\right]^2 x\,du \nonumber \\[0.5ex]
        &&\implies &\int \left[\frac{u}{\ln(3)}\right]^2 \,du \nonumber \\[0.5ex]
        &&\implies &\int \frac{u^2}{\ln^2(3)} \,du \nonumber \\[0.5ex]
        &&\implies &\frac{1}{\ln^2(3)}\int u^2 \,du \nonumber \\[0.5ex]
        &&\implies &\frac{1}{\ln^2(3)}\left[\frac{1}{3}u^3\right]+C \nonumber \\[0.5ex]
        &&\implies &\frac{u^3}{3\ln^2(3)}+C \nonumber \\[0.5ex]
        &&\implies &\boxed{\frac{\ln^3(x)}{3\ln^2(3)}+C}
    \end{alignat}

\end{subsubequations}

\subsection*{(b)}
\begin{subsubequations}
    {\Large \[ \int \frac{\left[ 2+\log_3(5x+2) \right]^5}{15x+6} \,dx \]}
    \[ u = 5x+2 \qquad du = 5\,dx \qquad dx = \frac{1}{5}\,du \]
    \begin{align}
        &\implies \int \frac{\left[ 2+\log_3(u) \right]^5}{3u}\cdot\frac{1}{5}\,du \\[1ex]
        &= \frac{1}{15} \int \frac{\left[2+\log_3(u)\right]^5}{u} \,du
    \end{align}
    \[ v = 2+\log_3(u) = 2+\frac{\ln(u)}{\ln(3)} \qquad dv = \frac{1}{\ln(3)} \cdot \frac{1}{u} \,du \qquad du = \ln(3)u\,dv \]
    \begin{align}
        &\implies \frac{1}{15} \int \frac{v^5}{u} \ln(3)u\,dv \\[1ex]
        &= \frac{\ln(3)}{15} \int v^5 \,dv \\[1ex]
        &= \frac{\ln(3)}{15} \cdot \frac{v^6}{6} + C \\[1ex]
        &= \frac{\ln(3)v^6}{80} + C \\[1ex]
        &= \boxed{ \frac{\ln(3)\left[2+\log_3(5x+2)\right]^6}{90} + C }
    \end{align}
\end{subsubequations}

\subsection*{(c)}
\begin{subsubequations}
    {\Large \[ \int 5x^{11} \sec(x^{12})\tan(x^{12})\,dx \]}    

    \begin{alignat}{2}
        & & & \nonumber \\
        &&&u = x^{12} \quad du = 12x^{11}dx \quad dx = \frac{du}{12x^{11}} \\
        &&\implies &\int 5x^{11}\sec(u)\tan(u) \,\frac{du}{12x^{11}} \\
        &&\implies &\frac{5}{12}\int \sec(u)\tan(u) \,du \\
        &&\implies &\frac{5}{12}\left[\sec(u)\right]+C \\
        &&\implies &\boxed{\frac{5}{12}\sec(x^{12})+C}
    \end{alignat}
\end{subsubequations}

\subsection*{(d)}
\begin{subsubequations}
    {\Large \[ \int \frac{1}{\sqrt{x}\,e^{-\sqrt{x}}}\sec^2(1+\sqrt{x}) \,dx \]}
    \[ u = \sqrt{x} \qquad du = \frac{1}{2\sqrt{x}}\,dx \qquad dx = 2\sqrt{x}\,du = 2u\,du \]
    \begin{align}
        \implies &\int \frac{1}{ue^{-u}} \sec^2(1+u) \,2u\,du \\[1ex] 
        \implies &2\int \frac{1}{e^{-u}} \sec^2(1+u) \,du \\[1ex] 
        \implies &2\int e^u \sec^2(1+u) \,du \\[1ex] 
    \end{align}
    Now to do integration by parts:
    \[ v = e^u \qquad dv = e^u \qquad w = \int \sec^2(1+u) = \tan(1+u) \]
    \begin{align}
        \implies &e^u\tan(1+u) - \int e^u\sec^2(1+u) \,du
    \end{align}
    Let:
    \[ I = \int e^u \sec^2(1+u) \]
    \begin{align}
        I &= e^u\tan(1+u) - I \\[1ex]
        2I &= e^u\tan(1+u) \\[1ex]
        I &= \frac{1}{2}e^u\tan(1+u) \\[1ex]
    \end{align}
    Returning back to our integral:
    \begin{align}
        \int e^u\sec^2(1+u)\,du &= \frac{1}{2}e^u\tan(1+u) \\[1ex]
        \intertext{Bring back our coefficient of 2:}
        2\int e^u\sec^2(1+u)\,du &= 2\left(\frac{1}{2}e^u\tan(1+u)\right) \\
        2\int e^u\sec^2(1+u)\,du &= e^u\tan(1+u)
    \end{align}
    Thus,
    \begin{align}
        \int \frac{1}{\sqrt{x}\,e^{-\sqrt{x}}}\sec^2(1+\sqrt{x})\,dx &= e^u\tan(1+u) + C \\[1ex]
                                                                     &= e^{\sqrt{x}}\tan(1+\sqrt{x}) + C
    \end{align}
    \[ \boxed{ e^{\sqrt{x}}\tan(1+\sqrt{x}) + C} \]
    % \[ u = 1+\sqrt{x} \qquad du = \frac{1}{2\sqrt{x}}\,dx \qquad dx = 2\sqrt{x}\,du \]
    % \begin{align}
    %     &\implies \int \frac{1}{\sqrt{x}\,e^{-\sqrt{x}}} \sec^2(u) \, 2\sqrt{x}\,du \\[1ex]
    %     &\implies 2\int \frac{1}{e^{-\sqrt{x}}} \sec^2(u) \,du \\[1ex]
    %     &\implies 2\int e^{\sqrt{x}} \sec^2(u) \,du \\[1ex]
    % \end{align}
    % \[ 
    %     v = e^{\sqrt{x}} \qquad 
    %     dv = \frac{1}{2\sqrt{x}}e^{\sqrt{x}} \qquad 
    %     w = \int \sec^2(u) \,du = \tan(u) \qquad 
    %     dw = \sec^2(u)
    % \]
    % \begin{align}
    %     \implies \int e^{\sqrt{x}} \sec^2(u) \,du = e^{\sqrt{x}} \tan(u) - \int \frac{1}{2\sqrt{x}}e^{\sqrt{x}} \tan(u) \,du
    % \end{align}
    % \begin{align}
    %     &\Rightarrow \int \frac{1}{u\,e^{-u}}\sec^2(1+u) u\,du \\
    %     &= \int e^u \sec^2(1+u) \,du
    % \end{align}
    % Now to integrate by parts:
    % \[ v = e^u \quad dv = e^u \quad w = \int \sec^2(1+u) \,du = \tan(1+u) \quad dw = \sec^2(1+u)\]
    % \begin{align}
    %     &\Rightarrow e^u\tan(1+u) - \int e^u\sec^2(1+u) \,du \\[1ex]
    %     I &= \int e^u\sec^2(1+u) \,du \\[1ex]
    %     \implies I &= e^u\tan(1+u)-I \nonumber \\ 
    %     &\implies 2I = e^u\tan(1+u) \nonumber \\
    %     &\implies I = \frac{1}{2}e^u\tan(1+u)
    %     \intertext{Thus,}
    %     \int &e^u\sec^2(1+u) \,du = \frac{1}{2}e^u\tan(1+u) \\
    %     \int &e^\sqrt{x} \sec^2{1+\sqrt{x}} \,du 
    % \end{align}
\end{subsubequations}

\end{subequations}


\end{document}