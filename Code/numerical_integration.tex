\documentclass{article}

\usepackage[letterpaper, top=1in, bottom=1in, left=1in, right=1in]{geometry}
\usepackage{amsmath}

\title{\textbf{Numerical Integration:}\\Using Area Formulas to Integrate Non-Elementary Functions}
\author{Laith Toom}
\date{3/10/2023}

\begin{document}
\maketitle
\newpage
\tableofcontents
\newpage

\section{Introduction}
There are some functions that we cannot integrate with 
the techniques we have used thus far, so we can just use 
area formulas to approximate the area under the curve.
\section{Trapezoidal Sums}
We already touched upon this with \textbf{Riemann Sums}:
\[ \sum_{i=1}^n f(x_i)\Delta x_i \text{\quad where \quad} \Delta x_i = \frac{x_n-x_1}{n} \]
A more precise method of approximation would be to use \textbf{Trapezoidal Sums}.
Instead of using the area of rectangles, we use the area of trapezoids: 
\[ \text{Area of a Trapezoid:\quad} \frac{1}{2}(a+b)h \]
As a series, we get:
\begin{subequations}
\begin{gather}
    \sum_{i=1}^n \frac{1}{2}\Delta x_i(f(x_i)+f(x_{i+1})) \\
    \frac{1}{2}\sum_{i=1}^{n} \Delta x_i(f(x_i)+f(x_{i+1}))
    \intertext{Since the next terms will be $x_{i+1}$ and $x_{i+2}$, in which there will be 
    doubles of x-values between $x_i$ and $x_n$ inclusive, we will get a formula:}
    \boxed{\frac{1}{2}\Delta x_i (f(x_i) + 2f(x_{i+1}) + \cdots + 2f(x_{n-1}) + f(x_n))}
\end{gather}
\end{subequations}


\end{document}